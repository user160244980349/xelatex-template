%!TEX root = ../main.tex
\documentclass[../main]{subfiles}

\begin{document}

\newpage
\section{Примеры оформления}
\label{sec:subject_domain}

Пример раздела 1 уровня. Есть ряд полезных ссылок, если что-то не устраивает -- можно попробовать покопаться в них \cite{gost-docs, gost-repo, gost-examples, master-report-repo}.

\subsection{Примеры оформления элементов}
Пример раздела 2 уровня.

\subsubsection{Пример рисунка}
Пример раздела 3 уровня. Пример оформления рисунка \ref{fig:pics_management_del}.

\begin{figure}[H]
    \centering
    \ffigbox[\FBwidth]
    {\caption{Пример рисунка\label{fig:pics_management_del}}}
    {\includegraphics[width=.9\textwidth]{malevich.png}}
    \vspace{-\baselineskip}
\end{figure}

\subsubsection{Пример таблицы}
Пример раздела 3 уровня. Пример таблицы \ref{tab:iot_clusters}.

\begin{ltwrap}{2mm}{1}{\footnotesize}
    \begin{longtable}[H]{|C{.05\x}|M{.475\x}|M{.475\x}|}
        \caption{Пример таблицы\label{tab:iot_clusters}}\\\hline
        \multicolumn{1}{|H{.05\x}|}{№}
        & \multicolumn{1}{H{.475\x}|}{Заголовок} 
        & \multicolumn{1}{H{.475\x}|}{Заголовок}\\\hline
        \endfirsthead
        \caption*{Продолжение таблицы \ref{tab:iot_clusters}}\\\hline
        \multicolumn{1}{|H{.05\x}|}{№}
        & \multicolumn{1}{H{.475\x}|}{Заголовок} 
        & \multicolumn{1}{H{.475\x}|}{Заголовок}\\\hline
        \endhead
        \endfoot
        \endlastfoot
        0 &  &  \\\hline
        1 &  &  \\\hline
        2 &  &  \\\hline
        3 &  &  \\\hline
        4 &  &  \\\hline
        5 &  &  \\\hline
        6 &  &  \\\hline
        7 &  &  \\\hline
        8 &  &  \\\hline
        9 &  &  \\\hline
        10 &  &  \\\hline
        11 &  &  \\\hline
        12 &  &  \\\hline
        13 &  &  \\\hline
        14 &  &  \\\hline
    \end{longtable}
\end{ltwrap}

\subsubsection{Пример формулы}
Для расчета NPV необходима ставка дисконтирования, вычисленная по формуле (\ref{eq:wacc}) (WACC):
\begin{equation}
    \label{eq:wacc}
    R = r_s \cdot \frac{V_s}{V} + r_d \cdot \frac{V_d}{V} \cdot (1-T)\ ,\; R = 0 + 0{,}078 \cdot 1 \cdot (1-0{,}2) = 0{,}0624,
\end{equation}
\makebox[1.25cm]{где\hfill}$R$ -- стоимость капитала,\\
\makebox[1.25cm]{}$r_s$ -- ставка по собственному капиталу,\\
\makebox[1.25cm]{}$V_s$ -- величина собственного капитала,\\
\makebox[1.25cm]{}$V$ -- общая сумма капитала,\\
\makebox[1.25cm]{}$r_d$ -- ставка по заемному капиталу,\\
\makebox[1.25cm]{}$V_d$ -- величина заемного капитала,\\
\makebox[1.25cm]{}$T$ -- ставка налога на прибыль.

\subsubsection{Примеры перечислений}
Пример перечисления:
\begin{enumerate}
    \item раз,
    \item два,
    \item три.
\end{enumerate}

Пример перечисления:
\begin{itemize}
    \item раз,
    \item два,
    \item три.
\end{itemize}

\subsubsection{Примеры ссылок}
Примеры ссылок \cite{texbook, latex:companion, latex2e, knuth:1984, lesk:1977}, \cite{texbook, latex2e, lesk:1977}, \cite{texbook, latex2e, knuth:1984, lesk:1977}.

\end{document}