%!TEX root = ../main.tex
\documentclass[../main]{subfiles}

\begin{document}

\newpage
\section{Заголовок 1}
\label{sec:subject_domain}

Пример оформления рисунков \ref{fig:pics_management_del} и \ref{fig:pics_management_del}:

\begin{figure}[H]
    \centering
    \ffigbox[\FBwidth]
    {\caption{Пример удаления слоя\label{fig:pics_management_del}}}
    {\includegraphics[width=.9\textwidth]{malevich.png}}
    \vspace{-\baselineskip}
\end{figure}

\begin{figure}[H]
    \centering
    \ffigbox[\FBwidth]
    {\caption{Презентационный прототип интерфейса\label{fig:proto}}}
    {\includegraphics[width=.9\textwidth]{malevich.png}}
    \vspace{-\baselineskip}
\end{figure}

Пример перечисления:
\begin{enumerate}
    \item определение источника информации о различной IoT-продукции,
    \item отправка поискового запроса,
    \item получение результатов запроса (список IoT-продуктов),
    \item определение производителей IoT-продукции,
    \item поиск официальных сайтов производителей в сети интернет,
    \item поиск раздела <<политика безопасности>> на сайтах производителей,
    \item скачивание политик безопасности,
    \item очистка скачанных веб-документов от лишних элементов разметки,
    \item слияние тегов и оборачивание сырого текста,
    \item нормализация пунктуации и отступов,
    \item извлечение текста из тегов.
\end{enumerate}

Еще пример перечисления:
\begin{enumerate}
    \item определение источника информации о различной IoT-продукции,
    \item отправка поискового запроса,
    \item получение результатов запроса (список IoT-продуктов),
    \item определение производителей IoT-продукции,
    \item поиск официальных сайтов производителей в сети интернет,
    \item поиск раздела <<политика безопасности>> на сайтах производителей,
    \item скачивание политик безопасности,
    \item очистка скачанных веб-документов от лишних элементов разметки,
    \item слияние тегов и оборачивание сырого текста,
    \item нормализация пунктуации и отступов,
    \item извлечение текста из тегов.
\end{enumerate}

Пример формулы \ref{eq:pp}.
\begin{equation}
    \label{eq:pp}
    PP = M - \frac{\sum^{M}_{t=0}CF_t}{CF_{M+1}}\ ,\; PP = 3 - \frac{-2 841}{3 318} = 3{,}86\:\text{кварталов} = 0,965\:\text{года},
\end{equation}

Пример таблицы \ref{tab:iot_clusters}.
\begin{ltwrap}{2mm}{1}{\footnotesize}
    \begin{longtable}[H]{|C{.05\x}|M{.475\x}|M{.475\x}|}
        \caption{Пример таблицы\label{tab:iot_clusters}}\\\hline
        \multicolumn{1}{|H{.05\x}|}{№}
        & \multicolumn{1}{H{.475\x}|}{Заголовок} 
        & \multicolumn{1}{H{.475\x}|}{Заголовок}\\\hline
        \endfirsthead
        \caption*{Продолжение таблицы \ref{tab:iot_clusters}}\\\hline
        \multicolumn{1}{|H{.05\x}|}{№}
        & \multicolumn{1}{H{.475\x}|}{Заголовок} 
        & \multicolumn{1}{H{.475\x}|}{Заголовок}\\\hline
        \endhead
        \endfoot
        \endlastfoot
        0 &  &  \\\hline
        1 &  &  \\\hline
        2 &  &  \\\hline
        3 &  &  \\\hline
        4 &  &  \\\hline
        5 &  &  \\\hline
        6 &  &  \\\hline
        7 &  &  \\\hline
        8 &  &  \\\hline
        9 &  &  \\\hline
        10 &  &  \\\hline
        11 &  &  \\\hline
        12 &  &  \\\hline
        13 &  &  \\\hline
        14 &  &  \\\hline
    \end{longtable}
\end{ltwrap}

Примеры ссылок \cite{MDPI10, MDPI11, MDPI12, MDPI13} \cite{MDPI10, MDPI12, MDPI14, MDPI15}.

\section{Заголовок 2}
Пример раздела 1 уровня.

\subsection{Заголовок 2}
Пример раздела 2 уровня.

\subsubsection{Заголовок 2}
Пример раздела 3 уровня.

\subsubsection{Заголовок 2}
Пример раздела 3 уровня.

\subsection{Заголовок 2}
Пример раздела 2 уровня.

\section{Заголовок 3}
Пример раздела 1 уровня.

\section{Заголовок 4}
Пример раздела 1 уровня.


\end{document}